\section{Введение}
\subsection{Цели}
Цель данного документа -- описать онлайн-сервис по предоставлению индивидуальных музыкальных рекомендаций
на основе предпочтений пользователя,
с целью облегчить поиск подходящей для пользователя музыки.

% \subsection{Соглашения о терминах}
% \begin{tabular}{|l|l|}
%     Слушатель & Человек, слушающий музыку \\
% \end{tabular}

\subsection{Предполагаемая аудитория и последовательность восприятия}
Сервис предназначен для предоставления музыкальных рекомендаций для слушателей.

\subsection{Охват проекта}
Сервис предназначен упростить поиск музыки по вкусу для всех слушателей.
Охват сервиса -- десятки стран, включая Россию, Германию, Великобританию и США.

\subsection{Ссылки на источники}
\begin{itemize}
    \item \texttt{https://www.last.fm/about}
    \item \texttt{https://www.last.fm/ru/about}
\end{itemize}

\section{Общее описание}
\subsection{Видение продукта}
Сервис сохраняет информацию о прослушанных треках,
из которой выводит группы похожих слушателей
и предлагает слушателю треки, понравившиеся похожим слушателям.

\subsection{Функциональность продукта}
\begin{itemize}
    \item Онлайн-радиостанция.
    \item Добавление и поиск треков.
    \item Рекомендательная система.
\end{itemize}

\subsection{Классы и характеристики пользователей}
Пользователь получает рекомендации на основе его музыкальных предпочтений.
Существует два класса пользователей: обычный пользователь и премиум-пользователь.

Пользователи-гости имеют доступ к:
\begin{itemize}
    \item Странице регистрации
    \item Странице <<о проекте>>
\end{itemize}

Обычные пользователи имеют доступ к:
\begin{itemize}
    \item Рекомендациям треков на основе предпочтений или популярного сейчас
    \item Профилю, включающему
    \begin{itemize}
        \item Никнейм пользователя
        \item Аватар пользователя
        \item Количество прослушанных треков
        \item Количество прослушанных исполнителей
        \item Отчеты о прослушанной музыке за последнюю неделю и за последний год
        \item Библиотеку
        \begin{itemize}
            \item Прослушиваний
            \item Исполниетелей
            \item Альбомов
            \item Композици
        \end{itemize}
        \item Подписки
        \item Подписчиков
        \item Любимые композиции
        \item Список увлечений
        \item События
        \item Соседей
        \item Теги
    \end{itemize}
\end{itemize}

Премиум-пользователи также имеют доступ к:
\begin{itemize}
    \item Отчету о прослушанной музыке за последний месяц
    \item Архивам отчетов
    \item Дополнительной статистике прослушиваний
    \item Редактированию сохраненной музыки
    \item Виду с изображениями из библиотеки
    \item Полноэкранному воспроизведению
    \item Раннему доступу
    \item Музыкальным совпадениям (сравнение вкусов с другими пользователями)
    \item Мейнстримометру (сравнение вкусов с трендами)
    \item Истории тегов
\end{itemize}

Администраторы также имеют доступ к:
\begin{itemize}
    \item Добавлению треков
    \item Удалению треков
    \item Добавлению администраторов
    \item Удалению администраторов
    \item Удалению аккаунтов
\end{itemize}

\subsection{Среда функционирования продукта}
\begin{itemize}
    \item Централизованные серверы
    \item Веб-интерфейс для прослушивания музыки онлайн
    \item Клиентские приложения для:
    \begin{itemize}
        \item Windows
        \item GNU/Linux
        \item MacOS
        \item Android
        \item iOS
    \end{itemize}
\end{itemize}

\subsection{Рамки, ограничения, правила и стандарты}
\subsubsection{Рамки}
\begin{itemize}
    \item Юридические ограничения, лицензирование музыки
\end{itemize}

\subsubsection{Ограничения}
\begin{itemize}
    \item Использование распространенных протоколов передачи данных (HTTPS, WebSocket)
    \item Использование YouTube в качестве сервиса хостинга и передачи музыки
\end{itemize}

\subsubsection{Правила и стандарты}
\begin{itemize}
    \item Структура базы данных на серверах
    \item Протоколы коммуникации
    \item Критерии анализа пользовательских предпочтений
\end{itemize}

\subsection{Допущения и зависимости}
\begin{itemize}
    \item Необходим доступ к базе музыки, лицензированной для использования на радио.
    \item Функциональность требует авторизации.
    \item Необходима HTTPS-сертификация
\end{itemize}

\section{Функциональность системы}
\subsection{Описание и приоритет}
Сервис сохраняет информацию о прослушанных треках,
из которой выводит группы похожих слушателей
и предлагает слушателю треки, понравившиеся похожим слушателям.
Задача имеет высокий приоритет, т.к. сервис является ключевым для бизнеса.

\subsection{Процессы}
\begin{itemize}
    \item Регистрация
    \item Авторизация
    \item Прослушивание музыки пользователем
    \item Получение информации из профиля
    \item Заполнение профиля
    \item Сохранение композиции
    \begin{itemize}
        \item В закладки
        \item В избранное
    \end{itemize}
    \item Оплата премиум-подписки
    \item Анализ полученной информации и расчет предпочтений пользователя
    \item Фиксация предпочтений
    \item Выдача рекомендаций на основе предпочтений
    \item Добавление треков
    \item Удаление треков
    \item Добавление администраторов
    \item Удаление администраторов
    \item Удаление аккаунтов
\end{itemize}

\subsection{Требования}
\begin{longtable}{|l|l|}
    \hline
    № & \multicolumn{1}{|c|}{\textbf{Требование}} \\ \hline
    \endhead

    \multicolumn{2}{|c|}{\textbf{Функциональные требования}} \\ \hline

    \multicolumn{2}{|c|}{\textbf{Гости сайта}} \\ \hline
    FR1 & Сайт должен предоставлять возможность прочитать информацию о сайте \\ \hline
    FR2 & Сайт должен предоставлять возможность зарегистрироваться на сайте \\ \hline
    FR3 & Сайт должен предоставлять возможность войти в существующий аккаунт \\ \hline

    \multicolumn{2}{|c|}{\textbf{Обычные пользователи}} \\ \hline
    FR4 & Сайт должен выдавать музыкальные рекомендации \\ \hline
    FR5 & Музыкальные рекомендации должны быть основаны \\
    & на собранной о пользователе информации \\ \hline
    FR6 & Сайт должен предоставлять возможность прослушивания музыки \\ \hline
    FR7 & Сайт должен предоставлять возможность сохранения трека \\
    & в закладки или избранное \\ \hline
    FR8 & Сайт должен предоставлять возможность просмотра профиля пользователя \\ \hline
    FR9 & Сайт должен предоставлять возможность заполнения профиля пользователя \\ \hline
    FR10 & Сайт должен предоставлять возможность оплаты премиум-подписки \\ \hline
    FR11 & Сайт должен показывать контекстную рекламу \\ \hline
    FR12 & Сайт должен сохранять статистику о прослушанных треках \\ \hline

    \multicolumn{2}{|c|}{\textbf{Премиум-пользователи}} \\ \hline
    FR13 & Сайт должен предоставлять возможность отключения рекламы \\ \hline
    FR14 & Сайт должен предоставлять возможность просмотра статистики \\ \hline
    FR15 & Сайт должен предоставлять возможность сравнения \\
    & музыкальных предпочтений с другими пользователями \\ \hline
    FR16 & Сайт должен предоставлять возможность сравнения \\
    & музыкальных предпочтений с текущими трендами \\ \hline
    FR17 & Сайт должен предоставлять возможность просмотреть \\
    & историю музыкальных предпочтений \\ \hline

    \multicolumn{2}{|c|}{\textbf{Администраторы}} \\ \hline
    FR18 & Сайт должен предоставлять возможность добавления треков \\ \hline
    FR19 & Сайт должен предоставлять возможность удаления треков \\ \hline
    FR20 & Сайт должен предоставлять возможность добавления администраторов \\ \hline
    FR21 & Сайт должен предоставлять возможность удаления администраторов \\ \hline
    FR22 & Сайт должен предоставлять возможность удаления отдельных аккаунтов \\ \hline

    \multicolumn{2}{|c|}{\textbf{Нефункциональные требования}} \\ \hline
    NR1 & Кегль любого текста на сайте должен быть не меньше $12$ точек \\ \hline
    NR2 & Сайт должен поддерживать передачу звука \\
    & на скорости $96$ кб/с на пользователя \\ \hline
    NR3 & Время загрузки любой страницы не должно превышать $2$ секунды при \\
    & скорости интернет-соединения от $100$ МБ/с \\ \hline
    NR4 & Сайт должен выдерживать нагрузку в $250 000$ \\
    & активных пользователей одновременно \\ \hline
    NR5 & В качестве хостинга треков должен использоваться YouTube \\ \hline
    NR6 & Сервис должен быть доступен от $90\%$ времени \\ \hline
    NR7 & Технические сбои сервиса должны исправляться в течение $48$ часов \\ \hline
    NR8 & Пользовательские данные должны храниться в зашифрованном виде \\ \hline
    NR9 & Для проверки пароля должен использоваться криптостойкий хеш \\ \hline
\end{longtable}

\subsection{Атрибуты требований}
\begin{longtable}{|l|c|c|c|c|}
    \hline
    № & Приоритет & Трудоемкость & Риск & Стабильность \\ \hline
    \endhead
    FR1  & Важное       &  4 ч.ч. & & \\ \hline
    FR2  & Обязательное &  4 ч.ч. & & \\ \hline
    FR3  & Обязательное &  4 ч.ч. & & \\ \hline
    FR4  & Обязательное &  8 ч.ч. & & \\ \hline
    FR5  & Важное       & 16 ч.ч. & & \\ \hline
    FR6  & Обязательное &  8 ч.ч. & & \\ \hline
    FR7  & Важное       &  8 ч.ч. & & \\ \hline
    FR8  & Потенциально & 24 ч.ч. & & \\
         & возможное    &         & & \\ \hline
    FR9  & Потенциально & 24 ч.ч. & & \\
         & возможное    &         & & \\ \hline
    FR10 & Обязательное & 16 ч.ч. & & \\ \hline
    FR11 & Обязательное & 16 ч.ч. & & \\ \hline
    FR12 & Важное       &  8 ч.ч. & & \\ \hline
    FR13 & Обязательное &  8 ч.ч. & & \\ \hline
    FR14 & Важное       &  8 ч.ч. & & \\ \hline
    FR15 & Потенциально & 24 ч.ч. & & \\
         & возможное    &         & & \\ \hline
    FR16 & Потенциально & 24 ч.ч. & & \\
         & возможное    &         & & \\ \hline
    FR17 & Потенциально & 16 ч.ч. & & \\
         & возможное    &         & & \\ \hline
    FR18 & Обязательное &  8 ч.ч. & & \\ \hline
    FR19 & Потенциально &  8 ч.ч. & & \\
         & возможное    &         & & \\ \hline
    FR20 & Потенциально &  4 ч.ч. & & \\
         & возможное    &         & & \\ \hline
    FR21 & Потенциально &  4 ч.ч. & & \\
         & возможное    &         & & \\ \hline
    FR22 & Потенциально &  4 ч.ч. & & \\
         & возможное    &         & & \\ \hline
    NR1  & Обязательное & -       & & \\ \hline
    NR2  & Обязательное & -       & & \\ \hline
    NR3  & Важное       & -       & & \\ \hline
    NR4  & Потенциально & -       & & \\
         & возможное    &         & & \\ \hline
    NR5  & Обязательное & -       & & \\ \hline
    NR6  & Обязательное & -       & & \\ \hline
    NR7  & Обязательное & -       & & \\ \hline
    NR8  & Обязательное & -       & & \\ \hline
    NR9  & Обязательное & -       & & \\ \hline
\end{longtable}

\section{Прочее}
\subsection{Модели процессов}
