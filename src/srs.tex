\section{Введение}
\subsection{Цели}
Цель данного документа -- описать онлайн-сервис по предоставлению индивидуальных музыкальных рекомендаций
на основе предпочтений пользователя,
с целью облегчить поиск подходящей для пользователя музыки.

% \subsection{Соглашения о терминах}
% \begin{tabular}{|l|l|}
%     Слушатель & Человек, слушающий музыку \\
% \end{tabular}

\subsection{Предполагаемая аудитория и последовательность восприятия}
Сервис предназначен для предоставления музыкальных рекомендаций для слушателей.

\subsection{Охват проекта}
Сервис предназначен упростить поиск музыки по вкусу для всех слушателей.
Охват сервиса -- десятки стран, включая Россию, Германию, Великобританию и США.

\subsection{Ссылки на источники}
\begin{itemize}
    \item \texttt{https://www.last.fm/about}
    \item \texttt{https://www.last.fm/ru/about}
\end{itemize}

\section{Общее описание}
\subsection{Видение продукта}
Сервис сохраняет информацию о прослушанных треках,
из которой выводит группы похожих слушателей
и предлагает слушателю треки, понравившиеся похожим слушателям.

\subsection{Функциональность продукта}
\begin{itemize}
    \item Онлайн-радиостанция.
    \item Добавление и поиск треков.
    \item Рекомендательная система.
\end{itemize}

\subsection{Классы и характеристики пользователей}
Пользователь получает рекомендации на основе его музыкальных предпочтений.
Существует два класса пользователей: обычный пользователь и премиум-пользователь.

Обычные пользователи имеют доступ к:
\begin{itemize}
    \item Интернет-радио
    \item Плагину для сторонних плееров, записывающему список прослушанных треков
    \item Рекомендациям треков и событий на основе предпочтений
    \item Профилю, включающему
    \subitem Список последних прослушанных треков
    \subitem Количество прослушанных треков
    \subitem Индикатор степени совпадения музыкального вкуса с другим пользователем
    \subitem Никнейм пользователя
    \subitem Аватар пользователя
    \subitem Дату регистрации
    \item Страницам исполнителей, редактируемым по принципам вики
\end{itemize}

Премиум-пользователи также имеют доступ к личным сообщениям.

\subsection{Среда функционирования продукта}
\begin{itemize}
    \item Централизованные серверы
    \item Веб-интерфейс для прослушивания музыки онлайн
    \item Клиентские приложения для:
    \subitem Windows
    \subitem GNU/Linux
    \subitem MacOS
    \subitem Android
    \subitem iOS
\end{itemize}

\subsection{Рамки, ограничения, правила и стандарты}
\begin{itemize}
    \item Юридические ограничения, лицензирование музыки
    \item Структура базы данных на серверах
    \item Протоколы коммуникации клиентских приложений с серверами
    \item Критерии анализа пользовательских предпочтений
\end{itemize}

\subsection{Допущения и зависимости}
\begin{itemize}
    \item Необходим доступ к базе музыки, лицензированной для использования на радио.
    \item Функциональность требует авторизации.
\end{itemize}

\section{Функциональность системы}
\subsection{Описание и приоритет}
Сервис сохраняет информацию о прослушанных треках,
из которой выводит группы похожих слушателей
и предлагает слушателю треки, понравившиеся похожим слушателям.
Задача имеет высокий приоритет, т.к. сервис является ключевым для бизнеса.

\subsection{Процессы}
\begin{itemize}
    \item Авторизация
    \item Прослушивание музыки пользователем на сайте или в приложении
    \item Прослушивание музыки пользователем в стороннем приложении с установленным плагином
    \item Оплата премиум-подписки
    \item Отправка информации о прослушанном треке на сервер
    \item Анализ полученной информации и расчет предпочтений пользователя
    \item Фиксация предпочтений
    \item Выдача рекомендаций на основе предпочтений
\end{itemize}

\subsection{Функциональные требования}
\begin{itemize}
    \item Сохранение информации о прослушанных треках
    \subitem Приоритет -- высокий
    \subitem Трудоемкость -- низкая
    \subitem Стоимость -- низкая
    \subitem Оценка времени реализации -- 1 день
    \item Расчет предпочтений на основе полученной информации
    \subitem Приоритет -- средний
    \subitem Трудоемкость -- высокая
    \subitem Стоимость -- высокая
    \subitem Оценка времени реализации -- 6 месяцев (время на исследование и реализацию)
    \item Рекомендация треков на основе предпочтений
    \subitem Приоритет -- низкий
    \subitem Трудоемкость -- низкая
    \subitem Стоимость -- высокая
    \subitem Оценка времени реализации -- 1 месяц (основное исследование уже проведено при рассчете предпочтений)
    \item Возможность прослушивания рекомендованных треков
    \subitem Приоритет -- высокий
    \subitem Трудоемкость -- средняя
    \subitem Стоимость -- средняя
    \subitem Оценка времени реализации -- 1 неделя (реализация любого из интерфейсов)
\end{itemize}

\section{Требования к внешним интерфейсам}
\subsection{Интерфейсы пользователя (UX)}
\begin{itemize}
    \item Веб-страница для прослушивания музыки и получения рекомендаций онлайн
    \subitem Приоритет -- средний
    \subitem Трудоемкость -- средняя
    \subitem Стоимость -- средняя
    \subitem Оценка времени реализации -- 1 неделя (создание страницы, использующей уже существующие данные)
    \item Клиентские приложения
    \subitem Приоритет -- средний
    \subitem Трудоемкость -- средняя
    \subitem Стоимость -- средняя
    \subitem Оценка времени реализации -- 1 неделя (создание приложений, использующих уже существующие данные)
    \item Плагины для сторонних плееров
    \subitem Приоритет -- средний
    \subitem Трудоемкость -- средняя
    \subitem Стоимость -- средняя
    \subitem Оценка времени реализации -- 1 неделя (создание приложений, использующих уже существующие данные)
\end{itemize}

\subsection{Программные интерфейсы}
\begin{tabular}{ll}
    \hline
    \textbf{Используемое ПО} & \textbf{Описание} \\ \hline
    Операционная система & Кроссплатформенное ПО \\ \hline
    Веб-браузер & Браузер с поддержкой HTML5 и JavaScript \\ \hline
    Медиаплеер & Плеер с поддержкой плагинов \\ \hline
\end{tabular}

\subsection{Интерфейсы оборудования}
\begin{itemize}
    \item Возможность вывода звука и изображения
\end{itemize}

\subsection{Интерфейсы связи и коммуникации}
\begin{itemize}
    \item HTTP API для браузерного плеера, официальных приложений и плагинов
\end{itemize}

\section{Нефункциональные требования}
\subsection{Требования к производительности}
\begin{itemize}
    \item Передача непрерывного звукового потока на скорости 96 кб/с на слушателя
    \item Время загрузки страницы -- не более 200 мс при скорости интернет-соединения от 100 МБ/с
\end{itemize}

\subsection{Требования к сохранности данных}
При утере данных в результате сбоя или повреждения оборудования данные должны восстанавливаться из
резервной копии с восстановлением данных, записанных между снятием резервной копией и сбоем,
путем повторения транзакций за этот период.

\subsection{Требования к качеству программного обеспечения}
\begin{itemize}
    \item \textbf{Доступность.} Сервис должен быть доступен в течение как минимум 90\% времени
    \item \textbf{Поддержка.} Технические сбои сервиса должны оперативно исправляться
    \item \textbf{Практичность.} Сервис должен быть простым в использовании
\end{itemize}

\subsection{Требования к безопасности системы}
Разработанное ПО должно использовать защищенное хранилище данных,
надежно хранить пользовательские данные, в т.ч. информацию о платных подписках,
в зашифрованном виде и без сохранения паролей,
с использованием вместо этого криптостойкого хеширования.

\section{Прочее}
\subsection{Модели процессов}
