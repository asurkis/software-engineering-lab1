\bigskip

\textbf{Вывод:} в данной лабораторной работе я составлял список требований для онлайн-сервиса.
Для этого я использовал:
\begin{itemize}
    \item Software Requirements Specification -- шаблон для описания требований.
    Использовал в качестве шаблона.
    \item UML -- модель универсальных диаграмм.
    Использовал для создания диаграммы процессов.
\end{itemize}

SRS существует в основном для составления договоров и записи требований в таком виде,
чтобы на них можно было ссылаться.
Этот шаблон является избыточным в рамках данной лабораторной работы,
поскольку описывает именно договор, а не только систему.

В данном случае UML позволяет быстро определить и визуализировать права доступа различных людей
к частям системы.
Например, на диаграмме можно увидеть, что права пользователя включают все права гостя,
права премиум-пользователя -- все права обычного пользователя,
а права администратора -- все права премиум-пользователя,
без необходимости отдельно проводить связи от операции до каждого актора,
которому она доступна.

Однако слишком большое количество связей на одной диаграмме делает ее неудобной для восприятия.
Одно из возможных решений этой проблемы -- разбить систему на части,
но оно неприменимо в рамках данной работы,
когда диаграмму нужно составить по существующей системе.
Поэтому мне пришлось довести UML-диаграмму до минимально читаемого вида,
оставив ее при этом связной.

Прецеденты использования в данной лабораторной работе были тривиальными на уровне системы.
Однако некоторые из них можно разбить на большее количество ступеней,
например, прецедент 5 (оплата подписки) включает также взаимодействие платежной системы с банком,
но это выходит за границы системы.

Таким образом, формализация требований в виде диаграмм и прецедентов позволяет быстро и дешево
проработать основные функции системы без написания кода,
на основании этого можно составить архитектуру системы.

